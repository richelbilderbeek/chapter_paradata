% In the end, use a latex table generated in R by xtable
\begin{table}[ht]
\centering
\begin{tabular}{rlllll}
  \hline
  SNP       & Primary/Conditional hit & Cis\_Trans & Effect strength & SE   & P-value \\
  \hline
  rs1653281 & Primary                 & Trans     & 0.1             & 0.01 & 2.90E-09 \\
  rs2701020 & Primary                 & Trans     & 0.2             & 0.02 & 2.11E-10 \\
  rs3875965 & Primary                 & Trans     & -1.3            & 0.13 & 2.62E-11 \\
  rs4984576 & Primary                 & Trans     & 0.4             & 0.14 & 4.26E-12 \\
  rs5058124 & Primary                 & Cis       & -0.5            & 0.05 & 3.71E-13 \\
  \hline
\end{tabular}
\caption{
  SNP association.
  Effect strength = some measure of effect strength
  \richel{
    I adapted this from the SKAT biomarkers paper. I think the
    data will look different (e.g. the first column should have
    single or multiple SNPs, as their effects may only occur when combined)
  }
} 
\label{tab:snp_associations}
\end{table}
